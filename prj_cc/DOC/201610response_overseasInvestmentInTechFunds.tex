\documentclass[11pt,a4paper]{article} \usepackage[parfill]{parskip}
\usepackage{fancyhdr} \usepackage{etoolbox} \usepackage{graphicx}
\usepackage{hyperref} \usepackage[T1]{fontenc} \usepackage{color}
\usepackage{wrapfig} \usepackage{url} \usepackage{pdfpages}
\usepackage{xcolor} \usepackage{graphicx} \usepackage[style=long,
nonumberlist, acronym]{glossaries} \makeglossaries
\loadglsentries[acronym]{acros.tex} \usepackage[style=authoryear,mergedate=basic,backend=biber,urldate=long]{biblatex}
\bibliography{capitalControls.bib} \defbibheading{bibempty}{}
\setcounter{tocdepth}{1}


\title{Inputs on RBI's draft framework for according approval to
  Indian residents to invest in overseas tech funds} \author{Finance
  Research Group\\IGIDR}

\begin{document}
\maketitle
\tableofcontents

\newpage 


\section{Introduction}

On September 30, 2016, the \gls{rbi} published on its website a draft
framework for according approval to an Indian Party investing in
overseas Startups through an Overseas Technology Fund (\textsl{draft
  framework}) under the \citetitle{fema}, seeking public comments. The
draft framework (a) proposes to allow Indian residents to invest in
\glspl{otf} with the previous approval of the \gls{rbi}; and (b)
specifies the conditions that Indian residents must satisfy to be
granted such an approval. This note contains our comments and inputs
on the draft framework.

We submit that the draft framework suffers from four problems:

\begin{enumerate}
\item It lacks any economic rationale supporting a case for allowing
  Indian residents to invest in some kinds of funds, namely
  \glspl{otf} and not others.

\item It is inconsistent with the principles of public administration
  and rule of law.

\item It lacks clarity on basic concepts on which it hinges, namely,
  the concepts of \glspl{otf} and overseas start-ups.

\item It is excessively restrictive.
\end{enumerate}

We have elaborated our inputs in Section \ref{s:comments} of this
note.

% \subsection{Overview of the draft framework}
% The draft framework provides an eligibility criteria and enlists
% conditions required to be complied with for obtaining RBI approval
% for investing in OTFs.  Among other things, some of the elements of
% the Draft Framework are as follows:

% \begin{enumerate}
% \item Only listed Indian companies are eligible to invest under the
%   framework.
% \item A minimum net-worth requirement of Rs. 500 crore has been
%   placed upon the Indian company.
% \item The Indian company must have a net profit track-record in the
%   three years preceding the application for obtaining approval.
% \item The aggregate investment in OTF has been restricted to 400\%
%   of the net-worth of the Indian company or USD 500 million,
%   whichever is less.
% \item The Indian company is permitted to invest only in overseas
%   startups which are aligned with the core business of the Indian
%   company.
% \item Reporting requirements in the form of the Annual Performance
%   Report mentioned in FEMA 120 have been imposed with respect to the
%   investment made by the OTF.
% \end{enumerate}

% The following section contains our observations with respect to the
% Draft Framework.
 

\section{Inputs on the draft framework}\label{s:comments}

\begin{description}
\item[Absence of economic rationale]: There is no economic rationale
  supporting a case for allowing Indian residents to invest in some
  kinds of funds and not others. Interventions in the policy governing
  capital outflows, must be supported by a consistent economic
  rationale linked to identified market failures. Similarly,
  relaxations in such policy must be crafted in accordance with sound
  principles of public economics. Specifically, there are two problems
  with the draft framework in connection with the lack of an economic
  rationale.

  \begin{itemize}
  \item \textsl{No identified market failure}: \citetitle{fema120}
    lacks an identified market failure for imposing restrictions on
    capital outflows. The rules governing capital outflows differ
    depending on various factors having no connection with the market
    failures associated with capital outflows. For instance, the rules
    differ depending on the legal constitution of the Indian entity
    desiring to make the investment\footnote{For instance, the rules
      for offshore investment by individual Indian residents and
      proprietary concerns differ from the rules for offshore
      investment by companies. Similarly, the rules for offshore
      investment by listed companies differ from those governing
      offshore investment by unlisted companies.}, the instrument
    sought to be invested in\footnote{For instance, investments by
      Indian residents are categorised into investments in a joint
      venture or wholly owned subsidiary abroad or investment in
      unlisted debt or investment in listed securities. Different
      rules apply for each of these instruments. Thus, for instance,
      only listed Indian entities and mutual funds are permitted to
      invest in a limited set of listed securities abroad.}, and the
    sector in which investment is sought to be made\footnote{For
      instance, where the Indian resident proposes to invest in the
      financial sector abroad, it has to fulfill additional criteria
      such as having a profit track record, etc.}.
  
    By artifically distinguishing between investments in funds which
    it terms as \glspl{otf} and other funds, the draft framework
    perpetuates the abovementioned approach of imposing restrictions
    and relaxing them, without any economic rationale. In the given
    case, neither the restriction on Indian residents on investing in
    offshore funds nor the proposed relaxation allowing them to invest
    in \glspl{otf}, are linked to any market failure (or in the latter
    case, the absence of it).

  \item \textsl{Imposes residency based measures on capital flows}:
    The IMF has, in its paper titled \citetitle{imf_capitalOutflows}
    summarised the existing literature and understanding of capital
    flow measures in partially capital account convertible
    countries. On the basis of this and global experience in the
    administration of capital controls on outflows, it makes
    recommendations for designing a policy framework for governing
    capital outflows by countries that do not have a fully convertible
    capital account. The policy framework recommended under this paper
    is summarised below:

    \begin{enumerate}
    \item In countries that have substantially liberalised their
      capital account, capital outflows must be managed primarily with
      macro-economic and financial sector policies.
    \item Capital flow management measures on outflows can be
      considered in (i) crisis or near crisis conditions; or (ii) to
      provide breathing space while more fundamental policy adjustment
      is implemented. These are temporary in nature and must be lifted
      once the crisis is controlled.
    \item Even when capital flow measures are implemented, they must
      be \textsl{not} be residency-based. Examples of residency-based
      capital flow measures are measures on residents' investments and
      transfers abroad, such as limits on residents' investments in
      financial instruments abroad.
    \end{enumerate}

    Contrary to the evolved understanding on capital flow measures,
    the draft framework imposes residency-based measures such as
    allowing Indian residents to invest in units of \glspl{otf} and
    not in units of other kinds of funds, allowing listed companies to
    so invest and not others, and so on and so forth.

  \end{itemize}

\item[Inconsistent with the principles of public administration and
  rule of law]: The draft framework is fundamentally inconsistent with
  the rule of law on account of two main reasons:

  \begin{itemize}

  \item Ad-hocism: The draft framework gives the following reason for
    allowing Indian residents to invest in \glspl{otf}:

  \begin{quote}
    Reserve Bank has been receiving references from various Indian
    parties to invest in Overseas Technology Funds which in turn will
    further invest in overseas technology startups. Such proposals
    generally do not meet the eligibility norms for making the
    overseas direct investment under the automatic route .... It is
    proposed that the Reserve Bank will deal with such requests under
    the approval route ...
  \end{quote}

  The reason for relaxation is linked to repeated requests from Indian
  residents to invest in \glspl{otf}. It perpetuates the ad-hoc nature
  of relaxations that has pervaded the Indian regulatory framework
  governing capital controls in India.

\item Approval route mechanism: The draft framework perpetuates the
  approval route mechanism in \citetitle{fema120}.\footnote{For
    instance, an Indian resident proposing to make a financial
    commitment exceeding USD 1 billion, requires to obtain the prior
    approval of \gls{rbi}. Similarly, where an Indian party proposes
    to invest in a foreign entity, through a share swap, the approval
    of the Foreign Investment Promotion Board is required.}  A mature
  regulatory framework governing capital outflows should leave no
  scope for the exercise of discretion. The criteria for allowing or
  not allowing investment abroad must be clearly laid out in the
  law. Once an Indian resident satisfies such criteria, the investment
  must be allowed without having to approach any authority for
  approval. This reduces transaction costs of investing abroad as well
  as the scope for exercising ad-hoc discretion.

  By mandating Indian residents to approach the \gls{rbi} for approval
  for investing in \glspl{otf}, even where the Indian resident
  satisfies the criteria specified in the draft framework, the draft
  framework perpetuates the approval route mechanism. The rule must be
  straightforward: Indian residents who satisfy the requirements
  specified by \gls{rbi} may invest abroad, without having to approach
  the \gls{rbi} for approval.
\end{itemize}

\item[Lack of clarity]: In addition to the two substantive issues
  listed above, the draft framework suffers from several drafting
  deficiencies, which in turn, increases the scope of discretion and
  abuse:

  \begin{enumerate}
  \item Definition of \glspl{otf}: Although the primary purpose of the
    draft framework is to regulate investment in an \gls{otf}, it does
    not define the concept of an \gls{otf}. To the best of our
    knowledge, there is no globally accepted terminology called an
    \gls{otf}, even as per industr practice. Given that the proposal
    is entirely linked to allowing investment by Indian residents in
    \glspl{otf}, precisely defining an \gls{otf} becomes paramount for
    its uniform and objective operation and compliance.\footnote{This
      argument is without prejudice to our argument that a proposal to
      allow Indian residents to invest in some funds, and not others,
      lacks any economic rationale and must be substantially revised.}

  \item Definition of overseas startup: The draft framework states
    that the \gls{otf} shall invest in overseas technology
    startups. It further states that the definition to be accorded to
    the term \emph{startup} shall vary across jurisdictions based on
    the definition allotted to it in the various jurisdictions. It is
    submitted that the concept of start-up is generally not defined in
    any jurisdiction, including India. This adds to the vagueness and
    ambiguity of the draft framework.

    % \item Lack of clarity regarding caution list specification: In
    %   paragraph A sub part b. of the Draft Framework, while
    %   specifying that the Indian company must not be in the caution
    %   list of the RBI, the Draft Framework contain the words
    %   `\emph{for long overdue export bills}' within parentheses. It
    %   is unclear whether the sheer presence of the Indian company in
    %   the caution list itself disqualifies an Indian company, with
    %   specific reference to overdue export bills, or whether only
    %   when an Indian company is present in the caution list on
    %   account of overdue export bills will it be disqualified.

  \item Sources of investment: The draft framework states that only
    internal accruals or accruals from group or associate companies of
    the Indian company in India may be used for investing in
    \glspl{otf}, and that funds borrowed from the banking system shall
    not be permitted. Again, while bank borrowed funds have been
    disallowed from being invested in \glspl{otf}, borrowings raised
    through say, bond issuances have not been explicitly
    barred. According different treatment to substantively the same
    economic transaction, defies all economic rationale. Downward
    investment by an Indian resident of a loan taken from a bank
    should, if at all, be a matter of prudential regulation and not
    capital controls.

  \item Definition of core business: The draft framework specifies
    that the business of the entity in which the \gls{otf} makes
    investment, must be aligned with the \emph{core business} of the
    Indian resident. However, the term core business has not been
    defined. Whether this core business shall include only the primary
    objectives of the Indian company as mentioned in its Memorandum of
    Association or even the ancillary objectives may be included, is
    not specified. Further, it has also not been specified whether
    only the primary business of the foreign entity ought to be
    aligned with the core business of the Indian company or an Indian
    company may invest in an overseas startup as long as one of the
    many business activities of the startup are aligned with the core
    business of the Indian company. It is submitted that, being an
    important element of the application of the draft framework,
    clarity on these aspects is critical toward the uniform
    application of the draft framework.

  \item Some direct investments in \textsl{overseas startups} under
    automatic route: Paragraph B(vi) of the draft framework, states
    that `\emph{in case the Indian party holds more than 10\% direct
      stake in an overseas startup, UIN may be allotted by the AD bank
      under the automatic route.}'. It is unclear whether a 10\%
    direct investment in an overseas startup is under the automatic or
    approval route. If it is the former, then it is unclear why the
    same cannot be made under the automatic route under the existing
    \citetitle{fema120}, as investment in an operational entity would
    not trigger the approval route at all.
  \end{enumerate}

\item[Excessively restrictive]: The draft framework imposes several
  restrictions on Indian residents investing in \glspl{otf}, which are
  not backed by any economic rationale. Some of these restrictions are
  illustrated below:

  \begin{enumerate}
  \item The draft framework allows only listed companies to invest in
    \glspl{otf}. Again, this restriction is not backed by any economic
    rationale. Mandating separate rules for investment, depending on
    the constitution of the entity, is redundant.
  \item Indian companies which have ``long overdue export'' bills are
    disallowed from investing in \glspl{otf}. An investment in a
    foreign security is an investment decision just like an investment
    in an Indian security. In a mature market economy, long overdue
    export bills cannot be a ground for disallowing an Indian resident
    from investing abroad.
  \item The draft framework states that \glspl{otf} in which Indian
    residents have invested, shall invest only in those overseas
    technology startups that are aligned with the core business of the
    Indian investing entity. This wrongly assumes that the Indian
    resident will, at all times, be in a position to control the
    activity of the \gls{otf}. Given that an \gls{otf} may be a widely
    held fund, such a restriction would be incapable of monitoring or
    enforcing, both for the investing entity as well as the regulator
    itself.
  \end{enumerate}

  Investing abroad offers Indian investors, all residents of India,
  reduced risk through diversification of holdings. The question of
  timing of capital account liberalisation is not the subject of this
  note. However, even at this stage of limited capital account
  converitibility, there is scope for simplifying the regulatory
  framework governing capital outflows, streamlining it to reflect a
  consistent economic rationale and bringing in sound principles of
  rule of law in the administration of this framework. This need has
  been recognized by expert committees constituted by the
  Government.\footnote{\textsl{See}, for instance, the
    \citetitle{taraporeCommittee}, the \citetitle{wgfi} and the
    \citetitle{fslrc}}. For instance, the \citetitle{raghuramRajan},
  states:

\begin{quote}
  We also need to make it easier for our individuals and institutions
  to invest abroad. For individuals, the primary task may be to
  simplify procedures, and liberalize the kinds of assets and managers
  that can be invested in. For our institutions like pension funds, we
  have to convince various constituencies that a portfolio diversified
  across the world is safer than a portfolio concentrated only in
  India, and has better risk properties (for one, it retains value
  when the Indian economy suffers a downturn). Regulatory authorities
  then have to allow institutional portfolios to become broadly and
  internationally diversified.
\end{quote}

Despite this, even at this advanced stage of liberalisation, the
framework governing capital outflows is excessively restrictive. For
instance, it places restrictions on the kind of consideration that an
Indian resident may accept for her investment abroad, the kinds of
instruments that Indian residents may invest in, kinds of activities
and structures that the offshore entity in which an Indian resident
has invested may engage in, controls on sale of shares in offshore
companies held by Indian residents, etc.

Another element of central planning in the current regulatory
framework is the approval route under \citetitle{fema120}. Investments
by Indian residents abroad, which do not fall under any of the
specified categories in \citetitle{fema120}, are under the approval
route (i.e. mandated to obtain the approval of \gls{rbi}). The factors
which \gls{rbi} takes into consideration for granting approval, are
not supported by any economic rationale relating to capital
outflows. For instance, in considering such applications, \gls{rbi}
will consider the \textsl{prima facie} viability of the joint venture
in which the investment is proposed to be made, the contribution to
external trade and other benefits that will accrue to India through
such investment, financial position and business track record of the
Indian resident, etc. None of these factors have any link to any
market failure associated with capital outflows, nor are they
supported by any imminent crisis warranting such controls.

Investment by an Indian resident in any fund is a means of
diversification of her portfolio. By creating artificial distinctions
between \glspl{otf} and funds which are not \glspl{otf}, allowing only
listed entities with a profit track record and having a certain size
to invest in \glspl{otf}, mandating approval for making such
investment even when the specified criteria are satisfied, etc., the
draft framework perpetuates the overly restrictive and complicated
regulatory framework for capital outflows.


% \item \textbf{Monitoring foreign entity:} In paragraph B(iii) of the
%   Draft Framework, although the approval to invest is being provided
%   to the Indian company, a UIN is to be allotted to the OTF. It is
%   unclear as to the rationale for the said allotment. A UIN is
%   allotted to an entity to be able to identify a market player and
%   monitor it. There is no reasonable explanation which may be
%   inferred as to why or how the RBI desires to monitor a foreign
%   entity which is beyond its realm of control or supervision. It is
%   the Indian investor who is and can be regulated by RBI. We fail to
%   understand the need for a UIN to the OTF.
\end{description}



\section{Recommendations for improvising the draft framework}

In light of the above, we submit the following inputs for revising the
draft framework:

\begin{enumerate}
\item The draft framework must not create an artificial distinction
  between \glspl{otf} and funds which are not \glspl{otf}. There is no
  economic rationale supporting such a distinction. Indian residents
  must be given the benefit of diversifying their portfolios by
  allowing investment in funds managed and investing abroad,
  notwithstanding the kinds of companies that the funds pick.

\item The draft framework must not distinguish between investment by
  listed and unlisted Indian companies in funds managed and investing
  abroad. Also, it must move away from a central planning philosophy
  by allowing only Indian companies of a certain size to invest in
  funds managed and investing abroad. Thus, networth requirements and
  profit track record must not be determinants of whether the company
  will be allowed to diversify its portfolio by investing in a fund
  managed and investing abroad.

\item The draft framework must lay down the criteria for investing
  abroad by incorporating them in \citetitle{fema120}. Once these
  criteria are satisfied, the Indian resident must be able to make the
  capital outflow, without having to obtain the prior approval of
  \gls{rbi}.
\end{enumerate}



\printbibliography

\end{document}

